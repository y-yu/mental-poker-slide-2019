\usepackage{luatexja}
\usepackage[hiragino-pron, nfssonly, deluxe, expert]{luatexja-preset}
% \usepackage{pgfpages}
\usepackage{fontspec}
\usepackage{epigraph}
\usepackage{etoolbox}
\usepackage{tikz}
\usepackage{framed}
\usepackage{mathtools}
\usepackage{listings}
\usepackage{libertine}
\usepackage{bxcoloremoji}
\usepackage{xcolor}
% \usepackage{multirow}
\usepackage{diagbox}
\usepackage{caption}
% \usepackage{tikz-qtree}

\definecolor{links}{HTML}{2A1B81}
\hypersetup{colorlinks,linkcolor=,urlcolor=links}

\usetheme{Boadilla}
\usecolortheme{seahorse}
% \usefonttheme{serif}

\setbeamercolor{page number in head/foot}{bg=blue!10}
\makeatother
\setbeamertemplate{footline}
{
  \leavevmode%
  \hbox{%
    \begin{beamercolorbox}[wd=.4\paperwidth,ht=2.25ex,dp=1ex,center]{author in head/foot}%
      \usebeamerfont{author in head/foot}\insertshortauthor\hspace*{1ex}(\insertshortinstitute)
    \end{beamercolorbox}%
    \begin{beamercolorbox}[wd=.2\paperwidth,ht=2.25ex,dp=1ex,center]{title in head/foot}%
      \usebeamerfont{title in head/foot}\insertshorttitle
    \end{beamercolorbox}%
    \begin{beamercolorbox}[wd=.3\paperwidth,ht=2.25ex,dp=1ex,center]{date in head/foot}%
      \insertshortdate
    \end{beamercolorbox}%
    \begin{beamercolorbox}[wd=.1\paperwidth,ht=2.25ex,dp=1ex,center]{page number in head/foot}%
      \insertframenumber{} / \inserttotalframenumber\hspace*{1ex}
    \end{beamercolorbox}}%
  \vskip0pt%
}
\makeatletter

\beamertemplatenavigationsymbolsempty

\setbeamertemplate{bibliography item}{\insertbiblabel}
\setbeamersize{description width=1cm}
\setbeamertemplate{items}[circle]
\setbeamertemplate{section in toc}[circle]
\setbeamertemplate{subsection in toc}{%
  \leavevmode\leftskip=2em
  {%
    \usebeamerfont*{itemize item}%
    \usebeamercolor{subsection number projected}%
    \color{bg}%
    \raise1.25pt\hbox{\donotcoloroutermaths$\bullet$}}%
  \hskip1.5ex\inserttocsubsection\par}
\setbeamercolor{title}{bg=white}
\setbeamertemplate{title page}
{%
  \vbox{}
  \vfill
  \begingroup
    \centering
    \hrulefill
    \vskip1em\par
    \begin{beamercolorbox}[sep=8pt,center,shadow=false,rounded=true]{title}
      \usebeamerfont{title}\inserttitle\par%
      \ifx\insertsubtitle\@empty%
      \else%
        \vskip0.25em%
        {\usebeamerfont{subtitle}\usebeamercolor[fg]{subtitle}\insertsubtitle\par}%
      \fi%     
    \end{beamercolorbox}%
    \hrulefill
    \vskip1em\par
    \begin{beamercolorbox}[sep=8pt,center,shadow=false,rounded=true]{author}
      \usebeamerfont{author}\insertauthor
    \end{beamercolorbox}
    \begin{beamercolorbox}[sep=8pt,center,shadow=false,rounded=true]{institute}
      \usebeamerfont{institute}\insertinstitute
    \end{beamercolorbox}
    \begin{beamercolorbox}[sep=8pt,center,shadow=false,rounded=true]{date}
      \usebeamerfont{date}\insertdate
    \end{beamercolorbox}\vskip0.5em
    {\usebeamercolor[fg]{titlegraphic}\inserttitlegraphic\par}
  \endgroup
  \vfill
}
\setbeamertemplate{blocks}[rounded][shadow=false]
\setbeamertemplate{note page}{\pagecolor{yellow!5}\insertnote}


% ============ ここを消すとNote消える ================
% \mode<handout>{%
%   \setbeameroption{show notes on second screen=right}%
% }
% ============ ここを消すとNote消える ================


\renewcommand{\kanjifamilydefault}{\gtdefault}

\resetcounteronoverlays{lstlisting}
\definecolor{bluegray}{rgb}{0.4, 0.6, 0.8}
\DeclareCaptionFormat{listing}{{\color{bluegray}\lstlistingname}#2#3}
\captionsetup[lstlisting]{format=listing, font={footnotesize}}

\setmonofont[Ligatures=TeX]{CMU Typewriter Text}

\title[Mental Poker]{%
  {\bfseries\rmfamily\mcfamily\huge\scshape
    Mental Poker%
  }%
}
\author[Yoshimura Hikaru]{%
  \textsc{Yoshimura} Hikaru(吉村 優)
}
\date[September 17, 2019]{%
  スタディサプリ\textsc{English All Hands} \\
  \oldstylenums{September 17, 2019} \\
  {\scriptsize (\href{https://github.com/y-yu/mental-poker-slide-2019}{\texttt{y-yu/mental-poker-slide-2019@\GITAbrHash}})}%
}
\institute[Recruit Marketing Partners Co., Ltd.]{%
  Recruit Marketing Partners Co., Ltd.\\
  \href{mailto:yyu@mental.poker}{yyu@mental.poker}
}

\input{./lib/quotebox.tex}
\input{./lib/footnotemark.tex}
\input{./lib/ballon.tex}
\input{./lib/listings.tex}
\input{vc.tex}

\setbeamertemplate{items}[circle]

\newcommand\ballcircle[1]{%
  {%
    \usebeamercolor{enumerate item}%
    \tikzset{beameritem/.style={circle,inner sep=0,minimum size=2ex,text=enumerate item.bg,fill=enumerate item.fg,font=\footnotesize}}%
    \tikz[baseline=(n.base)]\node(n)[beameritem]{#1};%
  }
}
\newcommand\ballref[1]{%
  \ballcircle{\ref{#1}}
}

\newcommand\ce[1]{%
  \coloremoji{#1}
}

\newenvironment{notes}
  {%
    \begin{xlrbox}{NotesBox}
    \begin{minipage}{.95\textwidth}
    \small\rmfamily\mcfamily
    \begin{itemize}
    \setlength{\itemindent}{0em}
  }{%
    \end{itemize}
    \end{minipage}
    \end{xlrbox}
    \note{\theNotesBox}}

\makeatletter
\newsavebox\temp@simple@callout@box
\newcommand{\simplecallout}[3]{%
  \sbox{\temp@simple@callout@box}{\mbox{%
    \begin{tabular}{l}
      #3%
    \end{tabular}
  }}%
  \begin{center}%
    \begin{tikzpicture}%
      \calloutquote[width=1.05\wd\temp@simple@callout@box,position={(#1.5,-0.2)},fill=#2,rounded corners]{
        #3%
      }%
    \end{tikzpicture}%
  \end{center}
}
\makeatother

\begin{document}

\frame{\maketitle}

\begin{frame}
  \frametitle{目次}

  \tableofcontents
\end{frame}

\section{自己紹介}
\begin{frame}
  \frametitle{自己紹介}
  
  \begin{columns}
    \begin{column}{0.4\textwidth}
      \begin{center}
        \begin{figure}
          \includegraphics[width=0.8\textwidth]{img/bird.png}
        \end{figure}
      \end{center}
 
      \begin{table}[h]
        \begin{tabular}{ll}
          Twitter & \href{https://twitter.com/\_yyu\_}{@\_yyu\_} \\
          Qiita &  \href{https://qiita.com/yyu}{yyu} \\
          GitHub &  \href{https://github.com/y-yu}{y-yu} \\
        \end{tabular}
      \end{table}
    \end{column}
    \begin{column}{0.6\textwidth}
      \pause
      \begin{itemize}
        \item 筑波大学 情報科学類卒(学士)
        \item プログラム論理研究室
        \item<+-> \LaTeX とかScalaとか暗号とか量子コンピューターとか

        \item<+-> 今日の発表は
        \href{https://atnd.org/events/51236}{2014年の大学内LTの発表}をリバイズしたもの
      \end{itemize}
    \end{column}
  \end{columns}
\end{frame}

\section{オンラインポーカー}

\begin{frame}
  \frametitle{オンラインポーカー}

  \pause
  \begin{itemize}
    \item<+-> インターネットを利用した
    オンラインポーカーはさかんに行なわれている

    \item<+-> 一方でオンラインポーカーには
    \textbf{サーバープログラム}という審判が存在する

    \item<+-> このサーバーがカードをシャッフルしたり
    役の判定などをするため、サーバーが公平な前提で
    オンラインポーカーは公平となる
  \end{itemize}
\end{frame}

\begin{frame}
  \frametitle{オンラインポーカーへの疑惑}
  
  \simplecallout{-}{red!20}{このサーバーは本当に公平なのか\ce{😈}}

  \pause
  \begin{itemize}
    \item<+-> サーバーはその気になれば、意図した順番に山札を並べたり、
    あるユーザーの手札を他のユーザーへ教えたりできる
  \end{itemize}
  
  \uncover<+->{
    \simplecallout{+}{green!20}{サーバーなしでオンラインポーカーやるか!}
  }

  \uncover<+->{
    \simplecallout{-}{cyan!20}{%
      信頼できる第三者\textbf{なし}の公平なポーカー \\%
      {\hfill\LARGE ``Mental Poker''\hfill}
    }
  }

  \begin{itemize}
    \item<+-> RSA暗号の発明者であるシャミア・リベスト・エーデルマンによって
    1981年に発表された\cite{Shamir1981}
  \end{itemize}

  \uncover<+->{
    \simplecallout{+}{blue!20}{今日はコンピューターを使わずに\textbf{物理的な方法}で解説}
  }
\end{frame}

\section{Mental Pokerの準備}

\begin{frame}
  \frametitle{登場人物}

  \begin{columns}
    \begin{column}{0.48\textwidth}
      \emph{アリス(Alice)}

      \begin{figure}[h]
        \includegraphics[height=0.5\textheight]{img/alice.png}
      \end{figure}
    \end{column}
   
    \begin{column}{0.48\textwidth}
      \emph{ボブ(Bob)}

      \begin{figure}[h]
        \includegraphics[height=0.5\textheight]{img/bob.png}
      \end{figure}
    \end{column}
  \end{columns}

  \begin{itemize}
    \item 図ではアリスを``\texttt{A}''とし、
    またボブを``\texttt{B}''とする
  \end{itemize}
\end{frame}

\begin{frame}
  \frametitle{用意するもの}

  \pause
  \begin{itemize}
    \item<+-> トランプの\textbf{カード}52枚
    \begin{figure}[h]
      \includegraphics[width=0.6\textwidth]{img/cards.png}
    \end{figure}
   
    \item<+-> 外側からは区別できない\textbf{箱}を52個
    \begin{figure}[h]
      \includegraphics[width=0.6\textwidth]{img/boxes.png}
    \end{figure}
  \end{itemize}
\end{frame}

\begin{frame}
  \frametitle{用意するもの}

  \begin{itemize}
    \item<+-> アリスとボブ
    それぞれのプレイヤーについて\textbf{南京錠}を52個ずつ
    \begin{figure}[h]
      \includegraphics[width=0.6\textwidth]{img/padlocks.png}
    \end{figure}
    \begin{itemize}
      \item この南京錠は全て、アリス・ボブがそれぞれに持つ1つの鍵で開錠できる
    \end{itemize}    
  \end{itemize}
\end{frame}

\begin{frame}
  \frametitle{アリス・ボブのできること}

  \pause
  \begin{columns}
    \begin{column}{0.6\textwidth}
      \begin{itemize}
        \item<+-> 任意のカードをちょうど1枚だけ箱に入れる

        \item<+-> ちょうど1枚のカードを箱から取り出す

        \item<+-> 箱に南京錠をかける
        \begin{itemize}
          \item 南京錠は箱に任意の数つけることができる
        \end{itemize}
      \end{itemize}
    \end{column}
    \begin{column}{0.4\textwidth}
      \begin{figure}[h]
        \includegraphics[height=0.8\textheight]{img/we_can.png}
      \end{figure}
    \end{column}
  \end{columns}
\end{frame}

\begin{frame}
  \frametitle{アリス・ボブのできること}

  \begin{itemize}
    \item<+-> 彼らの鍵を使って南京錠を取り外す
    \begin{figure}[h]
      \includegraphics[width=0.7\textwidth]{img/we_can2.png}
    \end{figure}
    \begin{itemize}
      \item<+-> 南京錠が箱に複数ついている場合、
      どのような順番で開錠しても箱の中身は変化しない
    \end{itemize}
  \end{itemize}
\end{frame}

\begin{frame}
  \frametitle{アリス・ボブのできないこと}

  \pause
  \begin{itemize}
    \item<+-> \textbf{あいていない箱}の中のカードを知る
    \begin{itemize}
      \item 箱の中にあるカードの情報を知るには、
      まず箱をあける必要がある
    \end{itemize}
   
    \item<+-> あいていない箱にカードを入れる
    \item<+-> あいていない箱からカードを取り出す
    
    \item<+-> \textbf{他者の南京錠が1つでもかかった箱}を開錠し、箱をあける
    \begin{figure}[h]
      \includegraphics[width=0.6\textwidth]{img/we_can_not.png}
    \end{figure}
  \end{itemize}
\end{frame}

\section{Mental Pokerのプロトコル}

\begin{frame}

  \centering
  {\huge Mental Pokerプロトコル}

  \vspace{1em}

  \begin{itemize}
    \item 山札づくり\footnote[frame]{%
      実はここから紹介するプロトコルは説明を簡単にするため僕が加えた変更により、
      アリスはチートができる。
      2014年の僕は気がつかなかったが、この発表を見ている人たちは気がつくだろうか?
    }

    {\color{gray!50}
      \item 手札のドロー
    }
  \end{itemize}
\end{frame}

\newcounter{ProtocolIndex}
\setcounter{ProtocolIndex}{1}
\newcommand*{\showIndex}{\theProtocolIndex .\;}
\newcommand*{\showAndIncrement}{%
  \showIndex%
  \stepcounter{ProtocolIndex}%
}

\begin{frame}
  \frametitle{\showAndIncrement アリスのターン}

  \begin{itemize}
    \item アリスは全ての箱に1枚ずつカードを入れ、全てにアリスの南京錠をかける
    \begin{figure}[h]
      \includegraphics[width=0.8\textwidth]{img/make_deck.png}
    \end{figure}
  \end{itemize}
\end{frame}

\begin{frame}
  \frametitle{\showIndex アリスのターン}

  \begin{itemize}
    \item<+-> アリスは全ての箱をボブへ送信する
    \begin{figure}[h]
      \includegraphics[width=0.9\textwidth]{img/send_to_b.png}
    \end{figure}

    \item<+-> このとき、アリスは自分で箱の中にカードを入れたので、
    カードと箱の順の対応を記録しておくことができる\ce{😈}
  \end{itemize}
\end{frame}

\stepcounter{ProtocolIndex}

\begin{frame}
  \frametitle{\showIndex ボブのターン}

  \begin{itemize}
    \item<+-> ボブは受け取った山札をシャッフルし、ボブの南京錠をかける
    \begin{figure}[h]
      \includegraphics[width=0.75\textwidth]{img/shuffle_deck.png}
    \end{figure}

    \item<+-> ボブはアリスがどのように箱を並べたのか知らないため、
    箱の中にあるカードについて情報を得ることができない

    \item<+-> またアリスは箱の順番を記録したかもしれないが、
    ボブによってシャッフルされたため分からなくなる\ce{😇}
  \end{itemize}
\end{frame}

\stepcounter{ProtocolIndex}

\begin{frame}
  \frametitle{\showIndex ボブのターン}

  \begin{itemize}
    \item<+-> ボブは全ての箱をアリスへ送信する
    \begin{figure}[h]
      \includegraphics[width=0.8\textwidth]{img/send_to_a.png}
    \end{figure}

    \item<+-> これで山札が完成\ce{🎉}
  \end{itemize}
\end{frame}

\stepcounter{ProtocolIndex}

\begin{frame}

  \centering
  {\huge Mental Pokerプロトコル}

  \vspace{1em}

  \begin{itemize}
    {\color{gray!50}
      \item 山札づくり
    }

    \item 手札のドロー%
    \footnote[frame]{この頃の僕はポーカーのルールである「テキサスホールデム」を理解していなかったので、手札を5枚ドローする}
  \end{itemize}
\end{frame}

\begin{frame}
  \frametitle{\showAndIncrement アリスのターン}

  \begin{itemize}
    \item アリスは全ての箱の中から5個を選び、それをボブへ送信する
    \begin{figure}[h]
      \includegraphics[width=0.9\textwidth]{img/select_boxes.png}
    \end{figure}
  \end{itemize}
\end{frame}

\begin{frame}
  \frametitle{\showIndex ボブのターン}

  \begin{itemize}
    \item<+-> ボブは受け取った5個の箱から、自分の南京錠をはずし
    箱をアリスへ送信する
    \begin{figure}[h]
      \includegraphics[width=0.8\textwidth]{img/unlock_b.png}
    \end{figure}
    \begin{itemize}
      \item<+-> このとき箱にはアリスの南京錠がまだ残っているため、
      ボブはこの5個の箱からカードを取り出すことができない
    \end{itemize}
  \end{itemize}
\end{frame}

\stepcounter{ProtocolIndex}

\begin{frame}
  \frametitle{\showIndex アリスのターン}

  \begin{itemize}
    \item<+-> アリスは受け取った5個の箱から、自分の南京錠をはずす
    \begin{figure}[h]
      \includegraphics[width=0.85\textwidth]{img/unlock_a.png}
    \end{figure}

    \item<+-> 箱にはもう南京錠がないため、5枚のカードを得る
  \end{itemize}
\end{frame}

\stepcounter{ProtocolIndex}

\begin{frame}
  \frametitle{\showIndex ボブのターン}

  \begin{itemize}
    \item<+-> 同様の手順で、ボブも5枚のカードを得る
    \begin{figure}[h]
      \includegraphics[width=0.85\textwidth]{img/turn_b.png}
    \end{figure}

    \item<+-> これで手札が完成\ce{🎉}
  \end{itemize}
\end{frame}

\stepcounter{ProtocolIndex}

\begin{frame}
  \frametitle{\showIndex アリス・ボブのターン}

  \begin{itemize}
    \item<+-> 後は普通にポーカーゲームをする
    \begin{itemize}
      \item カードを新たに山札からドローするときは、
      先程のプロトコルを行う
      \item カードをプレイヤー全員に公開するときは、
      単にカードを場に出せばよい
    \end{itemize}

    \item<+-> 誰かがコールするか1人を除いて全員がフォールドするなどして
    ゲームが終了したとき、参加者は全ての南京錠を開錠してカードが失くなったり、
    重複したりしていないかを確認する
    \begin{itemize}
      \item<+-> もし重複や紛失があった場合は不正とみなしゲームを無効とする
    \end{itemize}

    \item<+-> 次のゲームに進むときは、また山札作りプロトコルから
    やりなおす
  \end{itemize}
\end{frame}

\section{まとめ}

\begin{frame}
  \frametitle{まとめ}

  \pause
  \begin{itemize}
    \item<+-> このように公平な第三者なしでもポーカーができる
    \item<+-> シャッフル・ドローができれば実は多くのゲームを模倣できる
    \begin{itemize}
      \item たとえばサイコロは1から6までの数字のカードをシャッフルして1枚
      ドローする操作としてエンコードできる
    \end{itemize}

    \item<+-> 山札を最後に全て開錠しなくても不正が行われていないことを
    \textbf{ゼロ知識証明}で検証できる\cite{cmp}
    \begin{itemize}
      \item ゼロ知識証明は、たとえばいま数独パズルがあるとき、
      数独パズルの答えを誰かに教えることなく、
      自分が答えを知ってると証明する方法
    \end{itemize}

    \item<+-> 他にも\textbf{暗号通貨}と組み合せる研究\cite{Kumaresan}がある

    \item<+-> Mental Pokerを拡張してさらに色々な操作ができるようにした
    \textbf{秘密計算}は、機械学習と組み合せたりする応用\cite{ntt}がある
  \end{itemize}
\end{frame}

\section*{参考文献}

\begin{frame}[allowframebreaks]
  \frametitle{参考文献}

  \bibliographystyle{junsrt_url}
  %\nocite{*}
  \bibliography{ref}
\end{frame}

\begin{frame}
  \centering
  {\Huge Thank you for your attention!}
\end{frame}

\end{document}
